\chapter{Geschäftsmodelle}

In diesem Kapitel geht es um zwei unterschiedliche Geschäftsmodelle, welche während der Entwicklung des Produkts entstanden sind. Das erste Modell fokussiert dabei insbesondere den akademischen Sektor, während das zweite sich auf Onlinekurse konzentriert.

\section{Geschäftsmodell A: Freemium}
Von Beginn des Projekts an war Lightbulb Learning als Lern- und Prüfungsplattform vor allem für den akademischen Sektor, also für Universitäten und Fachhochschulen gedacht. Da die Zielgruppe das Geschäftsmodell wesentlich beeinflusst, besteht es aus der Monetarisierung durch die Professoren dieser Hochschulen bzw. im zweiten Schritt durch die Hochschulen selbst. Mit einem Freemium-Modell könnten so bestimmte Features oder eine begrenzte Kursgröße kostenfrei für den Einsteig zur Verfügung gestellt werden. Dieser könnte dann gegen einen festen monatlichen Preis um weitere Features und unbegrenzte Kursgrößen ergänzt werden, welcher pro Professor anfällt. Da das Wertversprechen eine echte Zeitersparnis der Professoren beinhaltet, galt die Annahme einer gewissen Zahlungsbereitschaft. Diese könnte entweder aus Budgets für digitale Werkzeuge der Professoren bestritten werden, im Zweifelsfall aber auch privat bezahlt werden. Würde ein bestimmter Schwellwert an Professoren der gleichen Hochschule eine Lightbulb Learning Lizenz beschaffen, so könnte eine ingesamt günstigere, hochschulweite Lizenz angeschafft werden, welche alle Premium Funktionen von Lightbulb Learning für alle Professoren und alle Studenten dieser Hochschule freischaltet. In dem Fall wäre auch eine Integration der Authentifizierungsschnittstelle denkbar, so dass die Zuordnung der Nutzer zu den echten Personen noch direkter ist. Durch die hochschulweite Premiumversion sollten auch diejenigen Professoren das Werkzeug kennenlernen, welche von sich aus keine Zahlungsbereitschaft gezeigt hätten, mit dem Ziel, weitere Professoren und Hochschulen von der Idee zu überzeugen, und somit einen Netzwerkeffekt in Gang zu setzen.

\subsection{Mindelsee-Stipendium}
Für die Finanzierung weiterer Zeit, in welcher weiter an Lightbulb Learning gearbeitet werden kann, wurde ein Businessplan für die Bewerbung beim Mindelsee-Stipendium erarbeitet. Dieser Businessplan enthält den zum Zeitpunkt der Bewerbung aktuellen Stand der Idee von Lightbulb Learning, eine Zielgruppen- und Marktanalyse, das Geschäftsmodell, die Finanzplanung und eine Chancen- und Risikoabwägung. Aus der Absage der Bewerbung konnte werthaltiges Feedback bezogen werden, woraus, insbesondere im Bezug auf das Geschäftsmodell, eine zweite Option entwickelt werden konnte. Da die Beschaffung von Software an Hochschulen teilweise bis auf Landesebene institutionell organisiert ist, sorgen die langen Entscheidungswege und die ausgiebige technologische Evaluation für eine gewisse Trägheit in der Adaption innovativer Software. Dies führt zur Erschwerung der Eindringung in diesen Markt, so dass die Risiken dieses Ansatzes zu hoch für eine ernsthafte Weiterverfolgung sind.

\section{Geschäftsmodell B: Affiliates-Programm}
In diesem Geschäftsmodell sind die primäre Zielgruppe, anstelle von Professoren und deren Studenten, Autoren von Onlinekursen und deren Teilnehmer. Ein Blick auf die aktuellen Zusammenhänge zwischen diesen Kursautoren und den Plattformen, auf denen sie ihre Kurse anbieten, eröffnet einen guten Zugang zum Mehrwert durch Lightbulb Learning. Einige Beispiele für Plattformen, auf denen Kursautoren ihre Kurse verkaufen, sind Udemy\footnote{\url{https://www.udemy.com/de/}}, Coursera\footnote{\url{https://www.coursera.org/}}, LinkedIn Learning\footnote{\url{https://www.linkedin.com/learning}} oder Academind\footnote{\url{https://academind.com/}}.

\subsection{Geschäftsmodell von Udemy}

In diesem Abschnitt soll eine genauere Untersuchung des Geschäftsmodells von Udemy erfolgen. Hier werden Kurse von Kunden gekauft und können daraufhin, in der Regel in erster Linie in Form von Videos, von diesen zu jedem beliebigen Zeitpunkt konsumiert werden. Die Anbieter der Kurse und die Betreiber der Plattform Udemy teilen sich dabei die Gewinne: ein Kursanbieter erhält 97\% der Gewinne nach allen Abzügen, wenn er einen Kursteilnehmer über einen Affiliatelink auf die Plattform bringt, während er 37\% erhält, wenn dies nicht nachvollzogen werden kann \cite{Udemy2022}. Das ist etwa dann der Fall, wenn jemand direkt über die Suche auf der Plattform einen Kurs kauft. Mit Abzüge beziehen sich dabei nicht nur auf die Mehrwertsteuer des jeweiligen Landes, sondern auch auf die Gebühr in Höhe von 30\% des Gesamtpreises, welche an Apple und Google bezahlt werden muss, wenn ein Kauf direkt über die jeweilige native App getätigt wird.

\subsection{Tests in Onlinekursen}
\label{sub:coursera}
Die Lernplattform Coursera erlaubt zusätzliche automatische Tests \cite[vgl.][]{Coursera2022quiz}, welche am Ende der jeweiligen Lektionen bestanden werden sollen, um zur nächsten Lektion fortzufahren. Dabei handelt es sich um automatisierte Multiple Choice Tests oder Kurzantwort-Quizfragen. Die pädagogische Fraglichkeit von Multiple Choice Aufgaben wurde bereits in Kapitel \ref{sec:eval} beschrieben, aber auch die Lösung der Kurzantwort-Quizfragen bringt Probleme mit sich. Unterscheiden sich die angegebenen Antworten in Wortwahl, Formulierung oder Grammatik, so kann diese nicht der korrekten Antwort zugeordnet werden und gilt somit als falsche Antwort. Durch diese Einschränkungen unterbietet die Usability der Kurzantwort-Quizfragen die pädagogische Sinnhaftigkeit. Ein Blogger kommentiert diese Tests mit: „So wie ich es verstanden habe, kannst du ein Quiz so oft wiederholen, bis du das Quiz bestanden hast. Manchmal musst du nach ein paar Versuchen jedoch eine gewisse Zeit warten, bevor du es erneut versuchen darfst. Die Lösungen von einigen Quizzen sind sogar schon im Internet zu finden” \cite{Geier2022}. Auch diese Aussage weist auf eine eher fragliche Bedeutung des Bestehens eines solchen Kurses hin.

\subsection{Ergänzung um Lightbulb Learning}
Die zweite Option für ein Geschäftsmodell für Lightbulb Learning besteht darin, den Autoren von Kursen zusätzlich zu ihren Einkünften auf den jeweiligen Plattformen Einkünfte durch die Vergabe von Zertifikaten auf Lightbulb Learning zu ermöglichen. Kursteilnehmer können einem vom Kursautor verwalteten Kurs kostenlos beitreten und ihre offenen Fragen, Antworten und gegenseitiges Feedback beitragen. Die Mechanismen des Crowdsourcing sorgen dabei für gegenseitige Auslöser und Moderation. Der Evaluationsmechanismus, der alle Beiträge eines Kursteilnehmers übersichtlich, kontextualisiert und chronologisch sortiert darstellt, unterstützt den Kursautor bei der Vergabe von Fortschritt in Form einer Prozentzahl auf Grundlage der geleisteten Beiträge. Erreicht ein Teilnehmer den Schwellwert der 100\%, so kann er zu einem vom Kursautor definierten Preis ein Zertifikat erhalten, welches in Form eines PDFs und als URL zur Verfügung gestellt wird. Der Erlös wird dann zwischen der Plattform und dem Kursautoren aufgeteilt. Dabei erhält der Kursautor einen höheren Prozentsatz als beispielsweise bei Udemy, so dass es eine wirtschaftlich kluge Entscheidung ist, aus Sicht eines Kursautors möglichst hochbepreisten Zertifikate auszugeben. Hypothetisch existiert aus Sicht der Kursautoren durch dieses Modell eine kausal zusammenhängende Kette von Mehrwerten, welche in den folgenden Abschnitten erörtert wird.

\subsection{Mehrwert I: Steigerung der Lernqualität}
Da die Komplexität, Adaptivität und Tiefe der Informationen, die man durch die Verwendung von Lightbulb Learning über das Thema des Onlinekurses diskutiert, größer ist, als wenn man die Informationen nur konsumiert, kann ein besseres Verständnis des Themas aufgebaut werden. Dies kann sich sowohl in der Leistung als auch in der Zufriedenheit der Teilnehmer, also sowohl objektiv als auch subjektiv, widerspiegeln. Da dies aus Sicht der Teilnehmer genau der Definition von Qualität des Kurses entspricht, folgen den Erfahrungen der Teilnehmer und deren Umfelds Weiterempfehlungen, welchen wiederum aus Anbietersicht eine Reichweitensteigerung folgt. Zusammenfassend lässt sich sagen, dass durch den Einsatz von Lightbulb Learning die Lernqualität eines Onlinekurses gesteigert werden kann.

\subsection{Mehrwert II: Evaluationssystem und Zertifikate}
\label{sub:certs}
Neben dem erhöhten Lerneffekt für die Teilnehmer und der daraus folgenden gesteigerten Qualität eines Kurses soll Lightbulb Learning ein System für die Evaluation der Teilnehmer ermöglichen. Besonders aktive Teilnehmer können von weniger aktiven Teilnehmern unterschieden werden, was eine quantitative Aussagekraft über die Leisungen der Teilnehmer impliziert. Zusätzlich kann der Kursautor sehr schnell einen qualitativen Eindruck gewinnen, indem die Beiträge jedes Teilnehmers chronologisch sortiert im Kontext des Dialogs dargestellt werden. Auf dieser Grundlage kann er bestimmten Teilnehmern ein Zertifikat ausstellen, welches die Kenntnis in dem betreffenden Bereich attestiert. Diese Funktion könnte perspektivisch sogar direkt von Lightbulb Learning angeboten werden, so dass der damit verbundene Aufwand aus Sicht des Anbieters minimiert wird. Das Zertifikat hat für beide Beteiligten einen besonderen Anreiz: Für den Teilnehmer bedeutet ein Zertifikat, dass er Gelerntes nicht nur anwenden, sondern auch nachweisen kann. Dies kann beispielsweise Fortschritte in der Karriere, die Abhebung von Konkurrenz, fachliche Umstiege oder Kundenzuwachs ermöglichen. Dem Autoren eines Onlinekurses bietet sich durch das Anbieten von Zertifikaten ebenfalls Mehrwerte unterschiedlicher Art. Erstens ist die Herausgabe eines Zertifikats ein Verkaufsargument, mit dem er sich und seinen Onlinekurs von einem anderen Kurs zu dem gleichen Thema abheben kann. Ein zweiter Vorteil ist die indirekte Folge von der Herausgabe von Zertifikaten an richtig evaluierte Teilnehmer, da diese durch ihr Zertifikat für den Kurs werben. Durch die Filterung qualifizierter Teilnehmer erhält das Zertifikat auf lange Sicht eine Wirkung, welche, wie bei einer starken Marke, für etwas wertvolles steht und somit dem Herausgeber des Zertifikats bei der Bildung eines guten Rufs nützt.

\subsection{Mehrwert III: Zusätzliche Einnahmequelle}
Der Wohl wichtigste Mehrwert und das stärkste Verkaufsargument für Anbieter von Online-Kursen ist die Absicherung der wirtschaftlichen Nachhaltigkeit. Mit Lightbulb Learning hat ein Anbieter die Möglichkeit, im Rahmen eines Affiliate-Programms, eine zusätzliche Einnahmequelle aufzubauen. Diese funktioniert so: Wird ein Kurs erstellt, der ab einer Evaluation von 100\% ein Zertifikat verspricht, so kann der Kursautor einstellen, wie viel dieses Zertifikat kosten soll. Die Wahl des Preises obliegt dem Kursanbieter. Der Bezahlvorgang wird von Lightbulb Learning abgewickelt, und der Kursanbieter erhält einen prozentualen Anteil des eingestellten Preises, beispielsweise 80\%. Somit kann der Anbieter seinen Teilnehmern gegenüber die Möglichkeit, ein Zertifikat zu erhalten, bewerben, sein Einkommen dadurch langfristig steigern, und die Abhängigkeit von der Preispolitik seiner aktuell verwendeten Lernplattform unabhängiger machen.

\subsection{Marketing und Vertrieb}
Das Geschäftsmodell B ist aus Perspektive von Lightbulb Learning auch deshalb interessant, weil die Eindringung in den Markt durch diesen Ansatz vereinfacht wird. Die zahlenden Kunden sind die Kursteilnehmer, so dass die Kursautoren ihr jeweiliges Publikum für die Teilnahme an den Kursen und den Erwerb der Zertifikate anwerben und dafür auch entsprechend incentiviert werden. Gegenüber den Kursautoren hingegen muss nicht verkauft werden, dass sie etwas für die Verwendung von Lightbulb Learning bezahlen müssen, sondern lediglich glaubhaft gemacht werden, dass die Verwendung eine zusätzliche Einnahmequelle und die beschriebenen zusätzlichen Mehrwerte bietet.
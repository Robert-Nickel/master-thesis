\chapter{Fazit}

In diesem Kapitel geht es darum, die Zusammenhänge zwischen den unterschiedlichen Bereichen herzustellen, und eine Einordnung in den Kontext der Frage zu machen, wie innovative digitale Produkte in der Zukunft entwickelt werden können. Zuletzt geht es um die Frage, ob es sich bei Lightbulb Learning um ein innovatives Produkt handelt.

\section{Lightbulb Learning, ein reaktives System?}
Das in reaktive Manifest beschreibt Prinzipien, welche zu robusteren, resilienteren, flexibleren und besser platzierteren Systemen führen \cite[vgl.][]{ReactiveManifesto}. Historisch ist dies in den Kontext der sich etablierenden Microservices-Architekturen, im Gegensatz zu den bis dahin üblichen, monolithisch-strukturierten Systemen, einzuordnen. Selbst frühe Cloud Anbieter wie AWS waren zu dem Zeitpunkt erst wenige Jahre alt und erfuhren starkes Wachstum, was als Alternative zu on-premise Systemen auf sogenannten bare metal Maschinen gesehen wurde. Auch die gesellschaftliche Einordnung ist wichtig: 2014 hatten Softwaresysteme durch die zunehmende Verfügbarkeit von Smartphones, Breitband-Internetanschlüssen und sich mit der Digitalisierung ändernde Geschäftsmodelle neue Anwendungsfälle, welche eine viel größere Skalierung, als es bisher nötig war, erforderten. Software musste immer und überall funktionieren, und bereits geringe Unterschiede in Latenzen oder Kompatibilitäten zu kleineren Geräten hatten wirtschaftlich weitreichende Folgen. Die im reaktiven Manifest beschriebenen Eigenschaften eines Systems beziehen sich auf diesen Kontext, der sich in den letzten 8 Jahren seinerseits weiterentwickelt hat. Mit Serverless-Technologie können viele der Probleme, die man durch die Anwendung des reaktiven Manifests lösen kann, ausgelagert und somit umgangen werden. Natürlich gibt es dafür Ausnahmen, Beispiele könnten aus den Bereichen des Machine Learnings oder des Internet-of-Things kommen. Mit Lightbulb Learning konnte jedoch ein Beispiel geschaffen welchem, mit welchem es möglich ist, ganz ohne eigene Verantwortung über die Infrastruktur ein stets antwortbereites und skalierbares System zu erstellen. Somit lässt sich zusammenfassen: Mit Lightbulb Learning werden nicht alle Eigenschaften reaktiver Systeme erfüllt, beispielsweise gibt es kein Konzept der Nachrichtenorientiertheit. Dennoch wurde das Ziel erfüllt, welches für reaktive Systeme definiert ist, nämlich durch Skalierbarkeit und Elastizität ein stets antwortbereites und resilientes System zu entwickeln. 

\section{Zusammenhang zwischen Lean Startup und Serverless-Technologie}
\label{sec:leansl}
Um ein Produkt nach den Lean Startup Prinzipien zu gründen, bedarf es nicht unbedingt der Serverless-Technologie. Dennoch lässt sich darin ein Fokus auf möglichst schlanke Experimente entdecken, welche durch den Einsatz dieser Gruppe an Technologien die Umsetzung solcher Experimente erleichtern. Die wesentlichen Eigenschaft von Serverless-Technologie ist dafür die Delegation von technologischer bzw. infrastruktureller Verantwortung an außenstehende Dienstleister. Zusammenfassend kann man also sagen, dass Serverless-Technologie für innovative Produkte nach Lean Startup zwar nicht unbedingt benötigt wird, es aber die Umsetzung dieser Ansätze vereinfacht.

\section{Zusammenhang zwischen agiler Entwicklung und Serverless-Technologie}
\label{sec:agilesl}
Eine ähnliche Beziehung besteht auch zwischen der agilen Entwicklung und Serverless-Technologie. Auch wenn methodologische Prinzipien an sich keine bestimmten Technologien voraussetzen, so erkennt man doch einige Parallelen in den zugrundeliegenden Prinzipien, woraus man schließen kann, dass Serverless-Technologie den Einsatz von agilen Prinzipien begünstigt. Ein konkretes Beispiel dafür ist die Unterteilung, Priorisierung und der Fokus der Arbeit auf fachliche Blöcke, und nicht auf technische, was sich auf dem ersten agilen Prinzip stützt, dass die höchste Priorität die Zufriedenstellung des Kunden durch frühe und kontinierliche Auslieferung wertvoller Software sei \cite{beck2001}. Da rein technische Schichten ohne die Umsetzung der fachlichen Anforderungen des Kunden dies nicht erfüllen können, arbeitet man stattdessen entlang der gewünschten Funktionalität. Eine Frage, welche in der Anwendung dieses Vorgehens häufig entsteht, ist die Rolle von technischen Abhängigkeiten. Ein Beispiel könnte lauten, dass für die Umsetzung von Feature A und Feature B ein neuer Microservice aufgesetzt werden muss, was mit einem bestimmten infrastrukturellen Aufwand verbunden ist. Wenn nicht klar ist, ob zuerst Feature A oder erst Feature B umgesetzt wird, was eine der wesentlichen Eigenschaften agiler Entwicklung ist, so ist es beispielsweise schwer, eine valide Schätzung der Komplexität der Umsetzung der jeweiligen Funktionen abzugeben. Die Gruppe solcher Probleme wird durch den Einsatz von Serverless-Technologie vereinfacht, da der Aufwand, der in solche technischen Themen investiert werden muss, insgesamt verringert wird. Es lässt sich also schließen, dass über die reine Kompatibilität zwischen den beiden Ansätzen hinaus der Einsatz von Serverless-Technologie den von agiler Entwicklung vereinfacht.

\section{Ist Serverless-Technologie die Zukunft der Apps?}
Es konnte gezeigt werden, dass Serverless-Technologie einige Eigenschaften erfüllt, welche typische Herausforderungen der Applikationsentwicklung adressiert. Der Trend in der Entwicklung von Software zu immer höheren Graden der Abstraktion von Technologie, um den Fokus auf die Geschäftslogik zu schärfen, setzt sich mit Serverless-Technologie fort. Die wichtige Frage wird bleiben, ob Serverless-Technologie die wichtigsten Probleme, die es bei der Entwicklung von digitalen Produkten gibt, adressiert. Durch die in den Abschnitten \ref{sec:leansl} und \ref{sec:agilesl} aufgezeigten Zusammenhänge lässt sich das zwar vermuten, da diese Frameworks in der nahen Vergangenheit einen großes Problem adressiert haben, allerdings könnten die großen Herausforderungen der Zukunft (in Hinsicht der Applikationsentwicklung) in dieser Frage auch aus Bereichen stammen, die zum gegenwärtigen Zeitpunkt nur schwer abgeschätzt werden können.

\section{Lightbulb Learning, eine Innovation?}
In diesem Abschnitt wird eruiert, ob es sich bei Lightbulb Learning tatsächlich um eine Innovation handelt. Der österreich-amerikanische Ökonom Joseph Alois Schumpeter definiert fünf Arten von Innovationen: Produktinnovationen, welche sich auf neuartige Produkte beziehen, Verfahrensinnovationen, welche durch neue Produktionsverfahren effizientere oder bessere Produkte anbieten und Marktinnovationen, welche durch die Erschließung neuer Absatzmärkte neue Einkommensströme generieren. Außerdem werden Beschaffungsinnovationen beschrieben, welche einen Wettbewerbsvorteil durch die Erschließung neuer Quellen bieten, sowie Strukturinnovationen, welche durch Neustrukturierung der Arbeitsorganisation effizientere Abläufe die daraus entstehenden besseren Produkte oder Dienstleistungen beschrieben. Eine Voraussetzung für eine echte Innovation ist außerdem die daraus erwachsende wirtschaftliche Wertschöpfung, welche sie von einer Erfindung unterscheidet \cite[vgl.][]{Sledzik2013}. Unter dieser Einschränkung kann Lightbulb Learning bisher lediglich den Anspruch erheben, eine potentielle Innovation zu sein, endgültig ist dies allerdings erst durch wirtschaftlichen Erfolg zu belegen. Dabei entspräche Lightbulb Learning einer Produktinnovation und insbesondere einer Verfahrensinnovation.

\subsection{Produktinnovation}
Die Idee hinter Lightbulb Learning, wie sie in Kapitel \ref{sec:idea} beschrieben wird, ist unter der genannten Einschränkung eine Produktinnovation, da zwei Eigenschaften von Leistungsbewertung im Bereich der Lernevaluation miteinander kombiniert werden konnten: Formativität und Skalierbarkeit.
\subsection{Verfahrensinnovation}
Um diese Produktinnovation zu implementieren, wurden innovative Werkzeuge aus dem Bereich der Serverless-Technologie verwendet. Diese ermöglichen eine Umsetzung von einer einzelnen Person in einem verhältnismäßig kurzen Zeitraum und somit einer deutlichen Geringhaltung der Kosten, was wiederum eine frühe Validierung ermöglicht. Ein Blick auf das gleiche Produkt mit anderer, etablierter Technologie verdeutlicht diese Aussage: Es hätte zusätzlich zum Frontend ein Backend entwickelt werden müssen, welches die Geschäftslogik beschreibt und separat getestet werden müsste. Der Betrieb des Backends, beispielsweise durch Containerisierung hätte übernommen werden müssen, und im Erfolgsfall müsste die Skalierung durch Orchestrierungswerkzeuge wie Kubernetes implementiert werden. Auch der Betrieb der Datenbank, die Skalierung und weitere infrastrukturelle Themen wie Backups müssten selbst übernommen und automatisiert oder zumindest verantwortet werden. Auch die Entwicklung einer eigenen Designsprache hätte weitere Kapazitäten und Kompetenzen erfordert, was zu einer größeren Abhängigkeit von Anderen und einer verlängerten Entwicklungszeit geführt hätte. Insgesamt hätte die Entwicklung dann ein Team von Entwicklern, Designern und Domänenexperten erfordert, was neben den gesteigerten Kosten auch die Komplexität effizienter Kommunikation und Kollaboration eingeführt hätte.\footnote{Sicherlich hätte die Entwicklung des Produkts als Team auch eine Reihe von Vorteilen, darauf liegt an dieser Stelle jedoch nicht der Fokus.}
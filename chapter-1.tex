\chapter{Einleitung}

In dieser Thesis geht es um die Skalierung einer Lern-Evaluationsplattform mithilfe von von Serverless-Technologie\footnote{Der Begriff Serverless lässt sich nicht präzise ins Deutsche übersetzen, da auch der Begriff Server üblicherweise nicht übersetzt wird.}. Der Begriff der Skalierung hat dabei drei wesentliche Bedeutungen. Erstens ist damit eine Skalierung der Lern-Evaluationsplattform als fachliche Lösung gemeint, welche auch für sehr große Kursgrößen funktionieren soll, ohne Einbüßung pädagogischer Sinnhaftigkeit. Die zweite Bedeutung ist die Skalierung von der Produktidee hin zu einem umfangreicher werdenden Produkt, mit einem Fokus auf wirtschaftliche und methodische Schwerpunkte. Drittens ist die Skalierung der Technologie gemeint, also die Sicherstellung, dass auch bei sehr vielen Zugriffen in kurzer Zeit das System vollständig antwortbereit bleibt. Die Betrachtung des Zusammenhang zwischen diesen drei Skalierungsdimensionen soll dabei helfen, ein vollständige Einordnung einiger Serverless-Technologien in den Kontext der Entwicklung von innovativen digitalen Produkten durchzuführen. Dafür geht eine Exploration einiger aktuell zur Verfügung stehenden Werkzeuge, deren Voraussetzungen und Potentiale sowie eine Kontextualisierung der Prinzipien im Vergleich zu üblichen Strukturen voran. Des Weiteren werden die Gemeinsamkeiten und Unterschiede zu reaktiven Systemen aufgezeigt. Um die jeweiligen Erkenntnisse greifbar zu machen, wird dazu unter Zuhilfenahme von Serverless-Technologie eine Lern-Evaluationsplattform namens Lightbulb Learning\footnote{\url{https://lightbulb-learning.io}} entwickelt. Anhand dieser Plattform können unterschiedliche Aspekte dieser Prinzipien konkret aufgezeigt werden. Inwiefern diese technologischen Prinzipien auch die angrenzenden Bereiche wie die Methodik und die Geschäftsmodell-Evaluation beeinflussen, wird ebenfalls untersucht. Das dabei entstehende Produkt soll nicht nur als Grundlage für diese Thesis dienen, sondern soll zusätzlich auch veröffentlicht werden. Dadurch soll eine genauere Einschätzung der Zusammenhänge der Produktentwicklung als Ganzes gewonnen werden können. Von der Serverless-Technologie selbst über die frühe Validierung der Funktionalität, die Entwicklungsgeschwindigkeit, die Aspekte der Qualitätssicherung und Vermarktung bis hin zu methodischen Prinzipien.
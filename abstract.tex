\chapter*{Abstract}
\setheader{Abstract}

This thesis is about scaling a learning-evaluation-platform, using serverless technology. The term scaling refers to three different meanings. First, the learning-evaluation-platform should scale to many students without losing its educational usefulness. Second, scaling means to develop a product idea through methodolical and economical guidelines to a product with an increasingly growing feature set. Third, the technology has to scale alongside the amount of users and requests, and always remain the state of responsiveness. Evaluating the set of serverless technologies, focussing on some of them in more detail, is the goal of this thesis. The focus of this investigation is the classification in the context of creating innovative digital products as a whole. Therefore, an examination of the currently available tools and frameworks is prefixed, their preconditions and potentials are evaluated, and a contextualisation of these principles compared to more established ones is made. The question of similarities and differences between serverless and reactive systems is answered. In order to make the results more practical, serverless technologies are used to build a learning-evaluation-platform called Lightbulb Learning\footnote{\url{https://lightbulb-learning.io}}, in order to show different aspects of these principles. Moreover, it is examined if and how the principles of serverless technology affect neighbouring topics such as the methodology of software development and business model evaluation. For a more complete assessment, that product should not only be the leading example of this thesis, but also be published and used by actual users. Thereby, even beyond technological aspects should be taken into consideration, so that the connection between serverless technology, development velocity, quality assurance, marketing and methodolical aspect can be understood. It shows, that serverless technology can support the implementation of concepts such as lean startup, agile development and design thinking by massively reducing the amount of technological preconditions in order to reach the state of having a working system. Moreover it shows, that scaling a serverless system requires less complexity through the abstraction of generic challenges in that field.